% Options for packages loaded elsewhere
\PassOptionsToPackage{unicode}{hyperref}
\PassOptionsToPackage{hyphens}{url}
%
\documentclass[
]{article}
\usepackage{amsmath,amssymb}
\usepackage{lmodern}
\usepackage{ifxetex,ifluatex}
\ifnum 0\ifxetex 1\fi\ifluatex 1\fi=0 % if pdftex
  \usepackage[T1]{fontenc}
  \usepackage[utf8]{inputenc}
  \usepackage{textcomp} % provide euro and other symbols
\else % if luatex or xetex
  \usepackage{unicode-math}
  \defaultfontfeatures{Scale=MatchLowercase}
  \defaultfontfeatures[\rmfamily]{Ligatures=TeX,Scale=1}
\fi
% Use upquote if available, for straight quotes in verbatim environments
\IfFileExists{upquote.sty}{\usepackage{upquote}}{}
\IfFileExists{microtype.sty}{% use microtype if available
  \usepackage[]{microtype}
  \UseMicrotypeSet[protrusion]{basicmath} % disable protrusion for tt fonts
}{}
\makeatletter
\@ifundefined{KOMAClassName}{% if non-KOMA class
  \IfFileExists{parskip.sty}{%
    \usepackage{parskip}
  }{% else
    \setlength{\parindent}{0pt}
    \setlength{\parskip}{6pt plus 2pt minus 1pt}}
}{% if KOMA class
  \KOMAoptions{parskip=half}}
\makeatother
\usepackage{xcolor}
\IfFileExists{xurl.sty}{\usepackage{xurl}}{} % add URL line breaks if available
\IfFileExists{bookmark.sty}{\usepackage{bookmark}}{\usepackage{hyperref}}
\hypersetup{
  pdftitle={Bioinformatics Lab 3},
  hidelinks,
  pdfcreator={LaTeX via pandoc}}
\urlstyle{same} % disable monospaced font for URLs
\usepackage[margin=1in]{geometry}
\usepackage{color}
\usepackage{fancyvrb}
\newcommand{\VerbBar}{|}
\newcommand{\VERB}{\Verb[commandchars=\\\{\}]}
\DefineVerbatimEnvironment{Highlighting}{Verbatim}{commandchars=\\\{\}}
% Add ',fontsize=\small' for more characters per line
\usepackage{framed}
\definecolor{shadecolor}{RGB}{248,248,248}
\newenvironment{Shaded}{\begin{snugshade}}{\end{snugshade}}
\newcommand{\AlertTok}[1]{\textcolor[rgb]{0.94,0.16,0.16}{#1}}
\newcommand{\AnnotationTok}[1]{\textcolor[rgb]{0.56,0.35,0.01}{\textbf{\textit{#1}}}}
\newcommand{\AttributeTok}[1]{\textcolor[rgb]{0.77,0.63,0.00}{#1}}
\newcommand{\BaseNTok}[1]{\textcolor[rgb]{0.00,0.00,0.81}{#1}}
\newcommand{\BuiltInTok}[1]{#1}
\newcommand{\CharTok}[1]{\textcolor[rgb]{0.31,0.60,0.02}{#1}}
\newcommand{\CommentTok}[1]{\textcolor[rgb]{0.56,0.35,0.01}{\textit{#1}}}
\newcommand{\CommentVarTok}[1]{\textcolor[rgb]{0.56,0.35,0.01}{\textbf{\textit{#1}}}}
\newcommand{\ConstantTok}[1]{\textcolor[rgb]{0.00,0.00,0.00}{#1}}
\newcommand{\ControlFlowTok}[1]{\textcolor[rgb]{0.13,0.29,0.53}{\textbf{#1}}}
\newcommand{\DataTypeTok}[1]{\textcolor[rgb]{0.13,0.29,0.53}{#1}}
\newcommand{\DecValTok}[1]{\textcolor[rgb]{0.00,0.00,0.81}{#1}}
\newcommand{\DocumentationTok}[1]{\textcolor[rgb]{0.56,0.35,0.01}{\textbf{\textit{#1}}}}
\newcommand{\ErrorTok}[1]{\textcolor[rgb]{0.64,0.00,0.00}{\textbf{#1}}}
\newcommand{\ExtensionTok}[1]{#1}
\newcommand{\FloatTok}[1]{\textcolor[rgb]{0.00,0.00,0.81}{#1}}
\newcommand{\FunctionTok}[1]{\textcolor[rgb]{0.00,0.00,0.00}{#1}}
\newcommand{\ImportTok}[1]{#1}
\newcommand{\InformationTok}[1]{\textcolor[rgb]{0.56,0.35,0.01}{\textbf{\textit{#1}}}}
\newcommand{\KeywordTok}[1]{\textcolor[rgb]{0.13,0.29,0.53}{\textbf{#1}}}
\newcommand{\NormalTok}[1]{#1}
\newcommand{\OperatorTok}[1]{\textcolor[rgb]{0.81,0.36,0.00}{\textbf{#1}}}
\newcommand{\OtherTok}[1]{\textcolor[rgb]{0.56,0.35,0.01}{#1}}
\newcommand{\PreprocessorTok}[1]{\textcolor[rgb]{0.56,0.35,0.01}{\textit{#1}}}
\newcommand{\RegionMarkerTok}[1]{#1}
\newcommand{\SpecialCharTok}[1]{\textcolor[rgb]{0.00,0.00,0.00}{#1}}
\newcommand{\SpecialStringTok}[1]{\textcolor[rgb]{0.31,0.60,0.02}{#1}}
\newcommand{\StringTok}[1]{\textcolor[rgb]{0.31,0.60,0.02}{#1}}
\newcommand{\VariableTok}[1]{\textcolor[rgb]{0.00,0.00,0.00}{#1}}
\newcommand{\VerbatimStringTok}[1]{\textcolor[rgb]{0.31,0.60,0.02}{#1}}
\newcommand{\WarningTok}[1]{\textcolor[rgb]{0.56,0.35,0.01}{\textbf{\textit{#1}}}}
\usepackage{graphicx}
\makeatletter
\def\maxwidth{\ifdim\Gin@nat@width>\linewidth\linewidth\else\Gin@nat@width\fi}
\def\maxheight{\ifdim\Gin@nat@height>\textheight\textheight\else\Gin@nat@height\fi}
\makeatother
% Scale images if necessary, so that they will not overflow the page
% margins by default, and it is still possible to overwrite the defaults
% using explicit options in \includegraphics[width, height, ...]{}
\setkeys{Gin}{width=\maxwidth,height=\maxheight,keepaspectratio}
% Set default figure placement to htbp
\makeatletter
\def\fps@figure{htbp}
\makeatother
\setlength{\emergencystretch}{3em} % prevent overfull lines
\providecommand{\tightlist}{%
  \setlength{\itemsep}{0pt}\setlength{\parskip}{0pt}}
\setcounter{secnumdepth}{-\maxdimen} % remove section numbering
\ifluatex
  \usepackage{selnolig}  % disable illegal ligatures
\fi

\title{Bioinformatics Lab 3}
\author{Biswas Kumar - Bisku859\\
Karthikeyan Devarajan - karde799}
\date{11/25/2021}

\begin{document}
\maketitle

\hypertarget{question-1}{%
\section{Question 1}\label{question-1}}

\hypertarget{question-1.1}{%
\subsection{Question 1.1}\label{question-1.1}}

\emph{Explain all the steps in the script required to obtain the
phylogeny and trait data.}\\
In step 1, A vector is created with accession numbers with z73494 and
AJ53426 to AJ534549.

\begin{Shaded}
\begin{Highlighting}[]
\NormalTok{x }\OtherTok{\textless{}{-}} \FunctionTok{paste}\NormalTok{(}\StringTok{"AJ5345"}\NormalTok{, }\DecValTok{26}\SpecialCharTok{:}\DecValTok{49}\NormalTok{, }\AttributeTok{sep =} \StringTok{""}\NormalTok{)}
\NormalTok{x }\OtherTok{\textless{}{-}} \FunctionTok{c}\NormalTok{(}\StringTok{"Z73494"}\NormalTok{, x)}
\end{Highlighting}
\end{Shaded}

Sequence are obtained using the accession number in GenBank database.

\begin{Shaded}
\begin{Highlighting}[]
\NormalTok{sylvia.seq }\OtherTok{\textless{}{-}} \FunctionTok{read.GenBank}\NormalTok{(x)}
\end{Highlighting}
\end{Shaded}

The list contains 25 sequences according to the accession number.
``AJ534549'' and ``AJ534548'' has 1041 nucleotides and remaining
sequences has 1143 nucleotides.

The alignments are done using cluster.

\begin{Shaded}
\begin{Highlighting}[]
\NormalTok{sylvia.clus }\OtherTok{\textless{}{-}} \FunctionTok{clustal}\NormalTok{(sylvia.seq)}
\end{Highlighting}
\end{Shaded}

The alignments can be done by mafft.

\begin{Shaded}
\begin{Highlighting}[]
\CommentTok{\#sylvia.maff \textless{}{-} mafft(sylvia.seq)}
\end{Highlighting}
\end{Shaded}

Code to save taxa.sylvia

\begin{Shaded}
\begin{Highlighting}[]
\NormalTok{taxa.sylvia }\OtherTok{\textless{}{-}} \FunctionTok{attr}\NormalTok{(sylvia.seq, }\StringTok{"species"}\NormalTok{)}
\FunctionTok{names}\NormalTok{(taxa.sylvia) }\OtherTok{\textless{}{-}} \FunctionTok{names}\NormalTok{(sylvia.seq)}
\NormalTok{taxa.sylvia[}\DecValTok{1}\NormalTok{] }\OtherTok{\textless{}{-}} \StringTok{"Sylvia\_atricapilla"}
\NormalTok{taxa.sylvia[}\DecValTok{24}\NormalTok{] }\OtherTok{\textless{}{-}} \StringTok{"Sylvia\_abyssinica"}
\end{Highlighting}
\end{Shaded}

create a nj distance matrix

\begin{Shaded}
\begin{Highlighting}[]
\NormalTok{f }\OtherTok{\textless{}{-}} \ControlFlowTok{function}\NormalTok{(xx) }\FunctionTok{root}\NormalTok{(}\FunctionTok{nj}\NormalTok{(}\FunctionTok{dist.dna}\NormalTok{(xx, }\AttributeTok{p=}\ConstantTok{TRUE}\NormalTok{)), }\StringTok{"AJ534526"}\NormalTok{)}
\end{Highlighting}
\end{Shaded}

Create a tree based on the distance matrix

\begin{Shaded}
\begin{Highlighting}[]
\NormalTok{tr }\OtherTok{\textless{}{-}} \FunctionTok{f}\NormalTok{(sylvia.clus)}
\NormalTok{nj.est }\OtherTok{\textless{}{-}}\NormalTok{ tr}
\end{Highlighting}
\end{Shaded}

plot the tree

\begin{Shaded}
\begin{Highlighting}[]
\NormalTok{nj.est}\SpecialCharTok{$}\NormalTok{tip.label }\OtherTok{\textless{}{-}}\NormalTok{ taxa.sylvia[tr}\SpecialCharTok{$}\NormalTok{tip.label]}
\FunctionTok{write.tree}\NormalTok{(nj.est, }\StringTok{"sylvia\_nj\_k80.tre"}\NormalTok{)}
\FunctionTok{plot}\NormalTok{(nj.est)}
\end{Highlighting}
\end{Shaded}

\includegraphics{lab-3_files/figure-latex/unnamed-chunk-8-1.pdf}

\begin{Shaded}
\begin{Highlighting}[]
\NormalTok{sylvia.eco }\OtherTok{\textless{}{-}} \FunctionTok{read.table}\NormalTok{(}\StringTok{"sylvia\_data.txt"}\NormalTok{)}
\end{Highlighting}
\end{Shaded}

Remove the out group species

\begin{Shaded}
\begin{Highlighting}[]
\NormalTok{nj.est }\OtherTok{\textless{}{-}} \FunctionTok{drop.tip}\NormalTok{(nj.est, }\StringTok{"Chamaea\_fasciata"}\NormalTok{)}
\end{Highlighting}
\end{Shaded}

\hypertarget{question-1.2}{%
\subsection{Question 1.2}\label{question-1.2}}

\emph{Analyze the discrete (type=discrete) geographical range variable
(DF\$geo.range) using ape::ace. Consider different models (parameter
model). Report on the results and interpret the estimated rates and
their standard errors.}

\begin{Shaded}
\begin{Highlighting}[]
\NormalTok{sylvia.eco }\OtherTok{\textless{}{-}} \FunctionTok{read.table}\NormalTok{(}\StringTok{"sylvia\_data.txt"}\NormalTok{)}
\NormalTok{DF }\OtherTok{\textless{}{-}}\NormalTok{ sylvia.eco[nj.est}\SpecialCharTok{$}\NormalTok{tip.label, ]}
\end{Highlighting}
\end{Shaded}

Model - SYM

\begin{Shaded}
\begin{Highlighting}[]
\CommentTok{\# SYM}
\NormalTok{model\_1 }\OtherTok{\textless{}{-}} \FunctionTok{ace}\NormalTok{(DF}\SpecialCharTok{$}\NormalTok{geo.range,}\AttributeTok{phy=}\NormalTok{nj.est,}\AttributeTok{type =} \StringTok{"discrete"}\NormalTok{,}\AttributeTok{model =} \StringTok{"SYM"}\NormalTok{)}
\FunctionTok{cat}\NormalTok{(}\StringTok{"The log likelihood for model is"}\NormalTok{,model\_1}\SpecialCharTok{$}\NormalTok{loglik,}\StringTok{"}\SpecialCharTok{\textbackslash{}n}\StringTok{"}\NormalTok{)}
\end{Highlighting}
\end{Shaded}

\begin{verbatim}
## The log likelihood for model is -16.97534
\end{verbatim}

\begin{Shaded}
\begin{Highlighting}[]
\FunctionTok{cat}\NormalTok{(}\StringTok{"The AIC value is "}\NormalTok{,}\FunctionTok{AIC}\NormalTok{(model\_1))}
\end{Highlighting}
\end{Shaded}

\begin{verbatim}
## The AIC value is  39.95068
\end{verbatim}

Model - ER

\begin{Shaded}
\begin{Highlighting}[]
\CommentTok{\# ER}
\NormalTok{model\_2 }\OtherTok{\textless{}{-}} \FunctionTok{ace}\NormalTok{(DF}\SpecialCharTok{$}\NormalTok{geo.range,}\AttributeTok{phy=}\NormalTok{nj.est,}\AttributeTok{type =} \StringTok{"discrete"}\NormalTok{,}\AttributeTok{model =} \StringTok{"ER"}\NormalTok{)}
\FunctionTok{cat}\NormalTok{(}\StringTok{"The log likelihood for model is"}\NormalTok{,model\_2}\SpecialCharTok{$}\NormalTok{loglik,}\StringTok{"}\SpecialCharTok{\textbackslash{}n}\StringTok{"}\NormalTok{)}
\end{Highlighting}
\end{Shaded}

\begin{verbatim}
## The log likelihood for model is -18.54616
\end{verbatim}

\begin{Shaded}
\begin{Highlighting}[]
\FunctionTok{cat}\NormalTok{(}\StringTok{"The AIC value is "}\NormalTok{,}\FunctionTok{AIC}\NormalTok{(model\_2))}
\end{Highlighting}
\end{Shaded}

\begin{verbatim}
## The AIC value is  39.09232
\end{verbatim}

The Symmetrical model could be good model since it has highest log
likelihood.

\hypertarget{question-2}{%
\section{Question 2}\label{question-2}}

Install the ade4 R package. Included with it you will find the carnivores
dataset, data(carni70).

\hypertarget{question-2.1}{%
\subsection{Question 2.1}\label{question-2.1}}

Explore the data set and report what can be found in it. Provide some
plots.

\begin{Shaded}
\begin{Highlighting}[]
\NormalTok{carni}\OtherTok{\textless{}{-}}\FunctionTok{data}\NormalTok{(carni70)}

\CommentTok{\#exploring dataset (list of 2 sets)}
\CommentTok{\#head(carni70)}

\NormalTok{carni\_tre}\OtherTok{\textless{}{-}}\NormalTok{carni70[}\DecValTok{1}\NormalTok{]}

\NormalTok{carni\_tab}\OtherTok{\textless{}{-}}\FunctionTok{as.data.frame}\NormalTok{(carni70[}\DecValTok{2}\NormalTok{])}

\CommentTok{\# Exploring the dataframe/tab through plots }

\CommentTok{\# scatterplot }
\FunctionTok{plot}\NormalTok{(carni\_tab}\SpecialCharTok{$}\NormalTok{tab.range,carni\_tab}\SpecialCharTok{$}\NormalTok{tab.size,}\AttributeTok{main =} \StringTok{"Scatterplot:Size Vs Range"}\NormalTok{,}\AttributeTok{xlab =} \StringTok{"Range"}\NormalTok{,}\AttributeTok{ylab=}\StringTok{"Size"}\NormalTok{,}\AttributeTok{col=}\StringTok{"red"}\NormalTok{)}
\end{Highlighting}
\end{Shaded}

\begin{center}\includegraphics{lab-3_files/figure-latex/unnamed-chunk-14-1} \end{center}

\begin{Shaded}
\begin{Highlighting}[]
\CommentTok{\#Density curve}
\FunctionTok{plot}\NormalTok{(}\FunctionTok{density}\NormalTok{(carni\_tab}\SpecialCharTok{$}\NormalTok{tab.size),}\AttributeTok{main=}\StringTok{"Density plot of Size"}\NormalTok{)}
\end{Highlighting}
\end{Shaded}

\begin{center}\includegraphics{lab-3_files/figure-latex/unnamed-chunk-14-2} \end{center}

\begin{Shaded}
\begin{Highlighting}[]
\FunctionTok{plot}\NormalTok{(}\FunctionTok{density}\NormalTok{(carni\_tab}\SpecialCharTok{$}\NormalTok{tab.range),}\AttributeTok{main=}\StringTok{"Density plot of Range"}\NormalTok{)}
\end{Highlighting}
\end{Shaded}

\begin{center}\includegraphics{lab-3_files/figure-latex/unnamed-chunk-14-3} \end{center}

\begin{Shaded}
\begin{Highlighting}[]
\NormalTok{scaled\_tab}\OtherTok{\textless{}{-}}\FunctionTok{scale}\NormalTok{(carni\_tab)}
\FunctionTok{print}\NormalTok{(}\StringTok{"The Heat map"}\NormalTok{)}
\end{Highlighting}
\end{Shaded}

\begin{verbatim}
## [1] "The Heat map"
\end{verbatim}

\begin{Shaded}
\begin{Highlighting}[]
\FunctionTok{heatmap}\NormalTok{(scaled\_tab)}
\end{Highlighting}
\end{Shaded}

\begin{center}\includegraphics{lab-3_files/figure-latex/unnamed-chunk-14-4} \end{center}

\begin{Shaded}
\begin{Highlighting}[]
\NormalTok{dist\_object}\OtherTok{\textless{}{-}}\FunctionTok{get\_dist}\NormalTok{(scaled\_tab)}

\CommentTok{\# Visualizing distance matrix}
\NormalTok{viz\_dist}\OtherTok{\textless{}{-}}\FunctionTok{fviz\_dist}\NormalTok{(dist\_object, }\AttributeTok{gradient =} \FunctionTok{list}\NormalTok{(}\AttributeTok{low =} \StringTok{"\#00AFBB"}\NormalTok{, }\AttributeTok{mid =} \StringTok{"white"}\NormalTok{, }\AttributeTok{high =} \StringTok{"\#FC4E07"}\NormalTok{))}

\NormalTok{viz\_dist}
\end{Highlighting}
\end{Shaded}

\begin{center}\includegraphics{lab-3_files/figure-latex/unnamed-chunk-14-5} \end{center}

\begin{Shaded}
\begin{Highlighting}[]
\CommentTok{\# Determining optimum number of cluster (elbow method), method = "wss" (for total within sum of square)}

\NormalTok{optimal\_cluster\_wss}\OtherTok{\textless{}{-}}\FunctionTok{fviz\_nbclust}\NormalTok{(scaled\_tab, kmeans, }\AttributeTok{method =} \StringTok{"wss"}\NormalTok{)}
\NormalTok{optimal\_cluster\_wss}
\end{Highlighting}
\end{Shaded}

\begin{center}\includegraphics{lab-3_files/figure-latex/unnamed-chunk-14-6} \end{center}

\begin{Shaded}
\begin{Highlighting}[]
\FunctionTok{cat}\NormalTok{(}\StringTok{"The Optimal number of cluser  with wss method : 3 }\SpecialCharTok{\textbackslash{}n}\StringTok{"}\NormalTok{)}
\end{Highlighting}
\end{Shaded}

\begin{verbatim}
## The Optimal number of cluser  with wss method : 3
\end{verbatim}

\begin{Shaded}
\begin{Highlighting}[]
\CommentTok{\# K means for 3 clusters }
\FunctionTok{set.seed}\NormalTok{(}\DecValTok{123}\NormalTok{)}
\NormalTok{k\_3 }\OtherTok{\textless{}{-}} \FunctionTok{kmeans}\NormalTok{(scaled\_tab, }\DecValTok{3}\NormalTok{, }\AttributeTok{nstart =} \DecValTok{25}\NormalTok{)}

\NormalTok{cluster\_grp\_no}\OtherTok{\textless{}{-}}\FunctionTok{table}\NormalTok{(k\_3}\SpecialCharTok{$}\NormalTok{cluster)}

\CommentTok{\# visualizing kmeans clusters }
\NormalTok{carni\_grp}\OtherTok{\textless{}{-}}\FunctionTok{fviz\_cluster}\NormalTok{(k\_3, }\AttributeTok{data =}\NormalTok{ scaled\_tab,}
             \AttributeTok{ggtheme =} \FunctionTok{theme\_minimal}\NormalTok{(),}
             \AttributeTok{main =} \StringTok{"Kmeans Clustering"}
\NormalTok{             )}


\CommentTok{\#building dataframe with centers and clusters}
\NormalTok{k\_3\_centers}\OtherTok{\textless{}{-}}\NormalTok{k\_3}\SpecialCharTok{$}\NormalTok{centers}

\NormalTok{cluster\_no }\OtherTok{\textless{}{-}} \FunctionTok{c}\NormalTok{(}\DecValTok{1}\SpecialCharTok{:} \DecValTok{3}\NormalTok{)}

\NormalTok{k\_3\_center\_df }\OtherTok{\textless{}{-}} \FunctionTok{data.frame}\NormalTok{(cluster\_no, k\_3\_centers)}

\FunctionTok{library}\NormalTok{(comprehenr)}
\NormalTok{No\_carnivores}\OtherTok{\textless{}{-}} \FunctionTok{to\_vec}\NormalTok{(}\ControlFlowTok{for}\NormalTok{(i }\ControlFlowTok{in} \DecValTok{1}\SpecialCharTok{:}\DecValTok{3}\NormalTok{) }\FunctionTok{table}\NormalTok{(k\_3}\SpecialCharTok{$}\NormalTok{cluster)[[i]])}
\NormalTok{K\_3\_cluster\_df }\OtherTok{\textless{}{-}} \FunctionTok{data.frame}\NormalTok{(cluster\_no, No\_carnivores)}

\CommentTok{\#frequency of carnivores in each K{-}3 cluster }
\NormalTok{K\_3\_cluster\_df}
\end{Highlighting}
\end{Shaded}

\begin{verbatim}
##   cluster_no No_carnivores
## 1          1            36
## 2          2            32
## 3          3             2
\end{verbatim}

Carni70 which is a data set of carnivores has a list of 2 sets (tre and
tab). tre is a character sets while tab is a data frame of 70
observations and 2 variables.On further inspection, it could be noticed
that tre is basically a phylogenetic tree whereas the tab data set are
quantitative traits.

The scatter and density plot was implemented to further understand the
distribution of data. Most of the carnivores on scatter plot have lower
size with two exceptions. These two exceptions could be the
outliers.Also,it could be easily seen that the density chart is heavily
skewed/ o weighted towards the left which means that most carnivores
have low size and only few are on the extreme.

The unsupervised k means clustering was implemented to check on the the
number of clusters using famous elbow method. We got three clusters, out
of which maximum data was distributed among first two cluster. It is
therefore cluster number 3 with 2 observations can be considered as out
liars as most of the data is concentrated between first two clusters

The heat map and distance matrix was also plotted above which depicts
two distinct group based on color patterns, that could be easily
visualized as two prominent clusters.

Moreover, the hierarchy (dendogram)in the chart could be cut in a way
that it gives two prominent cluster and a small cluster which could be
termed as outliers.

In short, we found two distinguish clusters in the given data-set with
few outliers.

\hypertarget{question-2.2}{%
\subsection{Question 2.2∗}\label{question-2.2}}

\emph{Analyze the two traits (size and range) with a number of different
phylogenetic comparative models. Look at the R packages ape, mvMORPH,
mvSLOUCH, ouch and slouch (the last one is NOT on CRAN and can be
downloaded from \url{https://github.com/kopperud/slouch}). Analyze the
data under the following models}\\
\emph{1. Both traits evolve as independent Brownian motions.}

\begin{Shaded}
\begin{Highlighting}[]
\CommentTok{\#1. Both traits evolve as independent Brownian motions, \# BM1 {-} No covariations}
\NormalTok{Carni\_tree }\OtherTok{\textless{}{-}}\FunctionTok{read.tree}\NormalTok{(}\AttributeTok{text =}\NormalTok{ carni70}\SpecialCharTok{$}\NormalTok{tre)}
\NormalTok{Brownian\_independent }\OtherTok{\textless{}{-}}\FunctionTok{mvBM}\NormalTok{(Carni\_tree, carni70}\SpecialCharTok{$}\NormalTok{tab,}\AttributeTok{error =} \ConstantTok{NULL}\NormalTok{, }\AttributeTok{model=}\StringTok{"BM1"}\NormalTok{,}\AttributeTok{param=}\FunctionTok{list}\NormalTok{(}\AttributeTok{constraint=}\StringTok{"diagonal"}\NormalTok{))}
\end{Highlighting}
\end{Shaded}

\begin{verbatim}
## row names of the data matrix must match tip names of your phylogeny! 
## successful convergence of the optimizer 
## a reliable solution has been reached 
## 
## -- Summary results for constrained rate  BM1 model -- 
## LogLikelihood:    -589.2373 
## AIC:      1186.475 
## AICc:     1186.771 
## 4 parameters 
## 
## Estimated rate matrix 
## ______________________ 
##           size    range
## size  73.70338  0.00000
## range  0.00000 13.54519
## 
## Estimated root state 
## ______________________ 
##            size    range
## theta: 38.43947 13.78439
\end{verbatim}

\begin{Shaded}
\begin{Highlighting}[]
\FunctionTok{summary}\NormalTok{(Brownian\_independent)}
\end{Highlighting}
\end{Shaded}

\begin{verbatim}
## mvMORPH model : BM1  summary 
## AIC : 1186.475 
## AICc: 1186.771 
## Log-Likelihood: -589.2373 
## Succesful convergence 
## Reliable solution
\end{verbatim}

\emph{2. The traits evolve as a correlated Brownian motion.}

\begin{Shaded}
\begin{Highlighting}[]
\CommentTok{\#2. Both traits evolve as  correlated Brownian motions}
\NormalTok{Brownian\_correlated }\OtherTok{\textless{}{-}}\FunctionTok{mvBM}\NormalTok{(Carni\_tree, carni70}\SpecialCharTok{$}\NormalTok{tab, }\AttributeTok{model=}\StringTok{"BM1"}\NormalTok{,}\AttributeTok{param=}\FunctionTok{list}\NormalTok{(}\AttributeTok{constraint=}\StringTok{"correlation"}\NormalTok{))}
\end{Highlighting}
\end{Shaded}

\begin{verbatim}
##  "shared","variance","correlation" and "proportional" can be used only with BMM model 
## row names of the data matrix must match tip names of your phylogeny! 
## successful convergence of the optimizer 
## a reliable solution has been reached 
## 
## -- Summary results for multiple rate BM1 model -- 
## LogLikelihood:    -588.7853 
## AIC:      1187.571 
## AICc:     1188.018 
## 5 parameters 
## 
## Estimated rate matrix 
## ______________________ 
##            size     range
## size  73.692486  3.578825
## range  3.578825 13.545170
## 
## Estimated root state 
## ______________________ 
##            size    range
## theta: 38.43947 13.78439
\end{verbatim}

\begin{Shaded}
\begin{Highlighting}[]
\FunctionTok{summary}\NormalTok{(Brownian\_correlated)}
\end{Highlighting}
\end{Shaded}

\begin{verbatim}
## mvMORPH model : BM1  summary 
## AIC : 1187.571 
## AICc: 1188.018 
## Log-Likelihood: -588.7853 
## Succesful convergence 
## Reliable solution
\end{verbatim}

\emph{3. Both traits evolve as independent Ornstein--Uhlenbeck
processes.}

\begin{Shaded}
\begin{Highlighting}[]
\CommentTok{\#3. Both traits evolve as independent Ornstein–Uhlenbeck processes}
\NormalTok{Carni\_tree\_a2o }\OtherTok{\textless{}{-}}\FunctionTok{ape2ouch}\NormalTok{(}\AttributeTok{tree =}\NormalTok{ Carni\_tree)}
\FunctionTok{summary}\NormalTok{(Carni\_tree\_a2o)}
\end{Highlighting}
\end{Shaded}

\begin{verbatim}
##   Length    Class     Mode 
##        1 ouchtree       S4
\end{verbatim}

\emph{4. The traits evolve as a bivariate Ornstein--Uhlenbeck process
(use mvMORPH or mvSLOUCH but be careful and check under what assumptions
the estimation is done).}

\begin{Shaded}
\begin{Highlighting}[]
\CommentTok{\#4. The traits evolve as a bivariate Ornstein–Uhlenbeck process }
\NormalTok{bivariate\_Ornstein }\OtherTok{\textless{}{-}}\FunctionTok{mvOU}\NormalTok{(}\AttributeTok{tree=}\NormalTok{Carni\_tree, }\AttributeTok{data=}\NormalTok{carni70}\SpecialCharTok{$}\NormalTok{tab, }\AttributeTok{model=}\StringTok{"OU1"}\NormalTok{,}\AttributeTok{diagnostic=}\ConstantTok{FALSE}\NormalTok{, }\AttributeTok{echo=}\ConstantTok{FALSE}\NormalTok{)}
\FunctionTok{summary}\NormalTok{(bivariate\_Ornstein)}
\end{Highlighting}
\end{Shaded}

\begin{verbatim}
## mvMORPH model : OU1  summary 
## AIC : 1161.213 
## AICc: 1162.312 
## Log-Likelihood: -572.6063 
## Succesful convergence 
## Reliable solution
\end{verbatim}

\emph{5. size evolves as a Brownian motion and range as an
Ornstein--Uhlenbeck process adapting to it (use slouch or mvSLOUCH and
be careful about column order).}

\begin{Shaded}
\begin{Highlighting}[]
\FunctionTok{rownames}\NormalTok{(carni70}\SpecialCharTok{$}\NormalTok{tab) }\OtherTok{\textless{}{-}}\NormalTok{ Carni\_tree}\SpecialCharTok{$}\NormalTok{tip.label}
\CommentTok{\#5. size evolves as a Brownian motion and range as an Ornstein–Uhlenbeck process adapting to it }
\NormalTok{Brownian\_motion }\OtherTok{\textless{}{-}} \FunctionTok{mvslouchModel}\NormalTok{(Carni\_tree,}\AttributeTok{mData=}\FunctionTok{as.matrix}\NormalTok{(carni70}\SpecialCharTok{$}\NormalTok{tab[,}\FunctionTok{c}\NormalTok{(}\DecValTok{2}\NormalTok{,}\DecValTok{1}\NormalTok{)]),}\AttributeTok{kY=}\DecValTok{1}\NormalTok{)}
\end{Highlighting}
\end{Shaded}

\begin{verbatim}
## Atype is at the default "Invertible" setting. This is a highly inefficient and unrecommended setting. Please look into the possible constraints on A and choose the one best corresponding to the hypothesis on the relationship between the traits.
\end{verbatim}

\emph{Compare the five models and say which one seems to fit better to the
data. Summarize (in your own words, NOT just provide the printout, as it
can be very long) the output under each model. If for some reason the
running time will be prohibitive (more than 12 hours) report how long
you ran the code and skip the model. Report on any problems you
experienced in your study.}

The five models under evaluation gives three metrics (AIC, AIC and
log-Likelihood) along with convergence status and reliability of
solution.

In order to compare, we will use the metrics to understand the
performance of the models.AIC score or AIC highlights the loss of
information.The minimum AIC value highlights the minimal loss of
information and hence selected as the best model among combinations.AIC
is used for smaller samples. On other hand, we look out for higher
log-likelihood for better performance of model.In the above results of
the models, we see that the AIC value is minimum for model no 4 i.e
bivariate Ornstein--Uhlenbeck model. It also have the highest
log-likelihood values. It is therefore, model no.4 (bivariate
Ornstein--Uhlenbeck model) performs the best among the received outputs.

\hypertarget{code-appendix}{%
\section{Code Appendix}\label{code-appendix}}

\begin{Shaded}
\begin{Highlighting}[]
\NormalTok{knitr}\SpecialCharTok{::}\NormalTok{opts\_chunk}\SpecialCharTok{$}\FunctionTok{set}\NormalTok{(}\AttributeTok{echo =} \ConstantTok{TRUE}\NormalTok{)}
\FunctionTok{library}\NormalTok{(ade4)}
\FunctionTok{library}\NormalTok{(tidyverse)}
\FunctionTok{library}\NormalTok{(phyloch)}
\FunctionTok{library}\NormalTok{(dplyr)}
\FunctionTok{library}\NormalTok{(stringr)}
\FunctionTok{library}\NormalTok{(data.table)}
\FunctionTok{library}\NormalTok{(factoextra)}
\FunctionTok{library}\NormalTok{(tibble)}
\FunctionTok{library}\NormalTok{(cluster)}
\FunctionTok{library}\NormalTok{(mvMORPH)}
\FunctionTok{library}\NormalTok{(ouch)}
\FunctionTok{library}\NormalTok{(abind)}
\FunctionTok{library}\NormalTok{(mvSLOUCH)}
\FunctionTok{library}\NormalTok{(ape)}
\NormalTok{x }\OtherTok{\textless{}{-}} \FunctionTok{paste}\NormalTok{(}\StringTok{"AJ5345"}\NormalTok{, }\DecValTok{26}\SpecialCharTok{:}\DecValTok{49}\NormalTok{, }\AttributeTok{sep =} \StringTok{""}\NormalTok{)}
\NormalTok{x }\OtherTok{\textless{}{-}} \FunctionTok{c}\NormalTok{(}\StringTok{"Z73494"}\NormalTok{, x)}
\NormalTok{sylvia.seq }\OtherTok{\textless{}{-}} \FunctionTok{read.GenBank}\NormalTok{(x)}
\NormalTok{sylvia.clus }\OtherTok{\textless{}{-}} \FunctionTok{clustal}\NormalTok{(sylvia.seq)}
\CommentTok{\#sylvia.maff \textless{}{-} mafft(sylvia.seq)}
\NormalTok{taxa.sylvia }\OtherTok{\textless{}{-}} \FunctionTok{attr}\NormalTok{(sylvia.seq, }\StringTok{"species"}\NormalTok{)}
\FunctionTok{names}\NormalTok{(taxa.sylvia) }\OtherTok{\textless{}{-}} \FunctionTok{names}\NormalTok{(sylvia.seq)}
\NormalTok{taxa.sylvia[}\DecValTok{1}\NormalTok{] }\OtherTok{\textless{}{-}} \StringTok{"Sylvia\_atricapilla"}
\NormalTok{taxa.sylvia[}\DecValTok{24}\NormalTok{] }\OtherTok{\textless{}{-}} \StringTok{"Sylvia\_abyssinica"}
\NormalTok{f }\OtherTok{\textless{}{-}} \ControlFlowTok{function}\NormalTok{(xx) }\FunctionTok{root}\NormalTok{(}\FunctionTok{nj}\NormalTok{(}\FunctionTok{dist.dna}\NormalTok{(xx, }\AttributeTok{p=}\ConstantTok{TRUE}\NormalTok{)), }\StringTok{"AJ534526"}\NormalTok{)}
\NormalTok{tr }\OtherTok{\textless{}{-}} \FunctionTok{f}\NormalTok{(sylvia.clus)}
\NormalTok{nj.est }\OtherTok{\textless{}{-}}\NormalTok{ tr}
\NormalTok{nj.est}\SpecialCharTok{$}\NormalTok{tip.label }\OtherTok{\textless{}{-}}\NormalTok{ taxa.sylvia[tr}\SpecialCharTok{$}\NormalTok{tip.label]}
\FunctionTok{write.tree}\NormalTok{(nj.est, }\StringTok{"sylvia\_nj\_k80.tre"}\NormalTok{)}
\FunctionTok{plot}\NormalTok{(nj.est)}
\NormalTok{sylvia.eco }\OtherTok{\textless{}{-}} \FunctionTok{read.table}\NormalTok{(}\StringTok{"sylvia\_data.txt"}\NormalTok{)}
\NormalTok{nj.est }\OtherTok{\textless{}{-}} \FunctionTok{drop.tip}\NormalTok{(nj.est, }\StringTok{"Chamaea\_fasciata"}\NormalTok{)}
\NormalTok{sylvia.eco }\OtherTok{\textless{}{-}} \FunctionTok{read.table}\NormalTok{(}\StringTok{"sylvia\_data.txt"}\NormalTok{)}
\NormalTok{DF }\OtherTok{\textless{}{-}}\NormalTok{ sylvia.eco[nj.est}\SpecialCharTok{$}\NormalTok{tip.label, ]}
\CommentTok{\# SYM}
\NormalTok{model\_1 }\OtherTok{\textless{}{-}} \FunctionTok{ace}\NormalTok{(DF}\SpecialCharTok{$}\NormalTok{geo.range,}\AttributeTok{phy=}\NormalTok{nj.est,}\AttributeTok{type =} \StringTok{"discrete"}\NormalTok{,}\AttributeTok{model =} \StringTok{"SYM"}\NormalTok{)}
\FunctionTok{cat}\NormalTok{(}\StringTok{"The log likelihood for model is"}\NormalTok{,model\_1}\SpecialCharTok{$}\NormalTok{loglik,}\StringTok{"}\SpecialCharTok{\textbackslash{}n}\StringTok{"}\NormalTok{)}
\FunctionTok{cat}\NormalTok{(}\StringTok{"The AIC value is "}\NormalTok{,}\FunctionTok{AIC}\NormalTok{(model\_1))}
\CommentTok{\# ER}
\NormalTok{model\_2 }\OtherTok{\textless{}{-}} \FunctionTok{ace}\NormalTok{(DF}\SpecialCharTok{$}\NormalTok{geo.range,}\AttributeTok{phy=}\NormalTok{nj.est,}\AttributeTok{type =} \StringTok{"discrete"}\NormalTok{,}\AttributeTok{model =} \StringTok{"ER"}\NormalTok{)}
\FunctionTok{cat}\NormalTok{(}\StringTok{"The log likelihood for model is"}\NormalTok{,model\_2}\SpecialCharTok{$}\NormalTok{loglik,}\StringTok{"}\SpecialCharTok{\textbackslash{}n}\StringTok{"}\NormalTok{)}
\FunctionTok{cat}\NormalTok{(}\StringTok{"The AIC value is "}\NormalTok{,}\FunctionTok{AIC}\NormalTok{(model\_2))}

\NormalTok{carni}\OtherTok{\textless{}{-}}\FunctionTok{data}\NormalTok{(carni70)}

\CommentTok{\#exploring dataset (list of 2 sets)}
\CommentTok{\#head(carni70)}

\NormalTok{carni\_tre}\OtherTok{\textless{}{-}}\NormalTok{carni70[}\DecValTok{1}\NormalTok{]}

\NormalTok{carni\_tab}\OtherTok{\textless{}{-}}\FunctionTok{as.data.frame}\NormalTok{(carni70[}\DecValTok{2}\NormalTok{])}

\CommentTok{\# Exploring the dataframe/tab through plots }

\CommentTok{\# scatterplot }
\FunctionTok{plot}\NormalTok{(carni\_tab}\SpecialCharTok{$}\NormalTok{tab.range,carni\_tab}\SpecialCharTok{$}\NormalTok{tab.size,}\AttributeTok{main =} \StringTok{"Scatterplot:Size Vs Range"}\NormalTok{,}\AttributeTok{xlab =} \StringTok{"Range"}\NormalTok{,}\AttributeTok{ylab=}\StringTok{"Size"}\NormalTok{,}\AttributeTok{col=}\StringTok{"red"}\NormalTok{)}

\CommentTok{\#Density curve}
\FunctionTok{plot}\NormalTok{(}\FunctionTok{density}\NormalTok{(carni\_tab}\SpecialCharTok{$}\NormalTok{tab.size),}\AttributeTok{main=}\StringTok{"Density plot of Size"}\NormalTok{)}
\FunctionTok{plot}\NormalTok{(}\FunctionTok{density}\NormalTok{(carni\_tab}\SpecialCharTok{$}\NormalTok{tab.range),}\AttributeTok{main=}\StringTok{"Density plot of Range"}\NormalTok{)}


\NormalTok{scaled\_tab}\OtherTok{\textless{}{-}}\FunctionTok{scale}\NormalTok{(carni\_tab)}
\FunctionTok{print}\NormalTok{(}\StringTok{"The Heat map"}\NormalTok{)}
\FunctionTok{heatmap}\NormalTok{(scaled\_tab)}


\NormalTok{dist\_object}\OtherTok{\textless{}{-}}\FunctionTok{get\_dist}\NormalTok{(scaled\_tab)}

\CommentTok{\# Visualizing distance matrix}
\NormalTok{viz\_dist}\OtherTok{\textless{}{-}}\FunctionTok{fviz\_dist}\NormalTok{(dist\_object, }\AttributeTok{gradient =} \FunctionTok{list}\NormalTok{(}\AttributeTok{low =} \StringTok{"\#00AFBB"}\NormalTok{, }\AttributeTok{mid =} \StringTok{"white"}\NormalTok{, }\AttributeTok{high =} \StringTok{"\#FC4E07"}\NormalTok{))}

\NormalTok{viz\_dist}

\CommentTok{\# Determining optimum number of cluster (elbow method), method = "wss" (for total within sum of square)}

\NormalTok{optimal\_cluster\_wss}\OtherTok{\textless{}{-}}\FunctionTok{fviz\_nbclust}\NormalTok{(scaled\_tab, kmeans, }\AttributeTok{method =} \StringTok{"wss"}\NormalTok{)}
\NormalTok{optimal\_cluster\_wss}
\FunctionTok{cat}\NormalTok{(}\StringTok{"The Optimal number of cluser  with wss method : 3 }\SpecialCharTok{\textbackslash{}n}\StringTok{"}\NormalTok{)}

\CommentTok{\# K means for 3 clusters }
\FunctionTok{set.seed}\NormalTok{(}\DecValTok{123}\NormalTok{)}
\NormalTok{k\_3 }\OtherTok{\textless{}{-}} \FunctionTok{kmeans}\NormalTok{(scaled\_tab, }\DecValTok{3}\NormalTok{, }\AttributeTok{nstart =} \DecValTok{25}\NormalTok{)}

\NormalTok{cluster\_grp\_no}\OtherTok{\textless{}{-}}\FunctionTok{table}\NormalTok{(k\_3}\SpecialCharTok{$}\NormalTok{cluster)}

\CommentTok{\# visualizing kmeans clusters }
\NormalTok{carni\_grp}\OtherTok{\textless{}{-}}\FunctionTok{fviz\_cluster}\NormalTok{(k\_3, }\AttributeTok{data =}\NormalTok{ scaled\_tab,}
             \AttributeTok{ggtheme =} \FunctionTok{theme\_minimal}\NormalTok{(),}
             \AttributeTok{main =} \StringTok{"Kmeans Clustering"}
\NormalTok{             )}


\CommentTok{\#building dataframe with centers and clusters}
\NormalTok{k\_3\_centers}\OtherTok{\textless{}{-}}\NormalTok{k\_3}\SpecialCharTok{$}\NormalTok{centers}

\NormalTok{cluster\_no }\OtherTok{\textless{}{-}} \FunctionTok{c}\NormalTok{(}\DecValTok{1}\SpecialCharTok{:} \DecValTok{3}\NormalTok{)}

\NormalTok{k\_3\_center\_df }\OtherTok{\textless{}{-}} \FunctionTok{data.frame}\NormalTok{(cluster\_no, k\_3\_centers)}

\FunctionTok{library}\NormalTok{(comprehenr)}
\NormalTok{No\_carnivores}\OtherTok{\textless{}{-}} \FunctionTok{to\_vec}\NormalTok{(}\ControlFlowTok{for}\NormalTok{(i }\ControlFlowTok{in} \DecValTok{1}\SpecialCharTok{:}\DecValTok{3}\NormalTok{) }\FunctionTok{table}\NormalTok{(k\_3}\SpecialCharTok{$}\NormalTok{cluster)[[i]])}
\NormalTok{K\_3\_cluster\_df }\OtherTok{\textless{}{-}} \FunctionTok{data.frame}\NormalTok{(cluster\_no, No\_carnivores)}

\CommentTok{\#frequency of carnivores in each K{-}3 cluster }
\NormalTok{K\_3\_cluster\_df}

\CommentTok{\#1. Both traits evolve as independent Brownian motions, \# BM1 {-} No covariations}
\NormalTok{Carni\_tree }\OtherTok{\textless{}{-}}\FunctionTok{read.tree}\NormalTok{(}\AttributeTok{text =}\NormalTok{ carni70}\SpecialCharTok{$}\NormalTok{tre)}
\NormalTok{Brownian\_independent }\OtherTok{\textless{}{-}}\FunctionTok{mvBM}\NormalTok{(Carni\_tree, carni70}\SpecialCharTok{$}\NormalTok{tab,}\AttributeTok{error =} \ConstantTok{NULL}\NormalTok{, }\AttributeTok{model=}\StringTok{"BM1"}\NormalTok{,}\AttributeTok{param=}\FunctionTok{list}\NormalTok{(}\AttributeTok{constraint=}\StringTok{"diagonal"}\NormalTok{))}
\FunctionTok{summary}\NormalTok{(Brownian\_independent)}
\CommentTok{\#2. Both traits evolve as  correlated Brownian motions}
\NormalTok{Brownian\_correlated }\OtherTok{\textless{}{-}}\FunctionTok{mvBM}\NormalTok{(Carni\_tree, carni70}\SpecialCharTok{$}\NormalTok{tab, }\AttributeTok{model=}\StringTok{"BM1"}\NormalTok{,}\AttributeTok{param=}\FunctionTok{list}\NormalTok{(}\AttributeTok{constraint=}\StringTok{"correlation"}\NormalTok{))}
\FunctionTok{summary}\NormalTok{(Brownian\_correlated)}
\CommentTok{\#3. Both traits evolve as independent Ornstein–Uhlenbeck processes}
\NormalTok{Carni\_tree\_a2o }\OtherTok{\textless{}{-}}\FunctionTok{ape2ouch}\NormalTok{(}\AttributeTok{tree =}\NormalTok{ Carni\_tree)}
\FunctionTok{summary}\NormalTok{(Carni\_tree\_a2o)}
\CommentTok{\#4. The traits evolve as a bivariate Ornstein–Uhlenbeck process }
\NormalTok{bivariate\_Ornstein }\OtherTok{\textless{}{-}}\FunctionTok{mvOU}\NormalTok{(}\AttributeTok{tree=}\NormalTok{Carni\_tree, }\AttributeTok{data=}\NormalTok{carni70}\SpecialCharTok{$}\NormalTok{tab, }\AttributeTok{model=}\StringTok{"OU1"}\NormalTok{,}\AttributeTok{diagnostic=}\ConstantTok{FALSE}\NormalTok{, }\AttributeTok{echo=}\ConstantTok{FALSE}\NormalTok{)}
\FunctionTok{summary}\NormalTok{(bivariate\_Ornstein)}

\FunctionTok{rownames}\NormalTok{(carni70}\SpecialCharTok{$}\NormalTok{tab) }\OtherTok{\textless{}{-}}\NormalTok{ Carni\_tree}\SpecialCharTok{$}\NormalTok{tip.label}
\CommentTok{\#5. size evolves as a Brownian motion and range as an Ornstein–Uhlenbeck process adapting to it }
\NormalTok{Brownian\_motion }\OtherTok{\textless{}{-}} \FunctionTok{mvslouchModel}\NormalTok{(Carni\_tree,}\AttributeTok{mData=}\FunctionTok{as.matrix}\NormalTok{(carni70}\SpecialCharTok{$}\NormalTok{tab[,}\FunctionTok{c}\NormalTok{(}\DecValTok{2}\NormalTok{,}\DecValTok{1}\NormalTok{)]),}\AttributeTok{kY=}\DecValTok{1}\NormalTok{)}
\end{Highlighting}
\end{Shaded}


\end{document}
